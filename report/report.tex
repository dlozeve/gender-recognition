\documentclass[a4paper,11pt,openany,extrafontsizes,oneside,article]{memoir}

%%\directlua{pdf.setminorversion(3)}

\usepackage{fontspec}

\setmainfont[Numbers=OldStyle]{Linux Libertine O}
\setsansfont[Numbers=OldStyle]{Linux Biolinum O}
% \setmonofont{Inconsolata}
\setmonofont[Scale=0.83]{Bitstream Vera Sans Mono}
% \usepackage{lmodern}

\usepackage{polyglossia}
\setdefaultlanguage{english}

\usepackage{graphicx}
\usepackage{xcolor}
%\usepackage{wrapfig}
\usepackage{subcaption}
%\usepackage{lettrine}
%\usepackage{tikz}
\usepackage{amsmath,amssymb}
\usepackage{booktabs}

% \usepackage{unicode-math}
% \setmathfont{texgyrepagella-math.otf}

%\usepackage{pdfpages}

\usepackage{microtype}

%% Propriétés du document PDF
\usepackage[unicode,colorlinks=true,linkcolor=blue]{hyperref}

\hypersetup{
  pdfauthor={P272},
  pdftitle={Statistical Machine Learning: Assessed Practical},
  pdfsubject={Assessed Practical},
  pdfkeywords={stats,practical,report},
  pdfpagemode=UseOutlines,
  pdfpagelayout=TwoColumnRight
}

\renewcommand{\chapterautorefname}{section}

\definecolor{bg}{rgb}{0.95,0.95,0.95}

\usepackage{minted}
\setminted[r]{autogobble,fontsize=\footnotesize}

%% Pour la classe memoir /!\

%% Marges
\setulmarginsandblock{3cm}{3cm}{*}
\setlrmarginsandblock{3cm}{3cm}{*}
\checkandfixthelayout%

%% Numérotation des divisions logiques
\setsecnumdepth{subsection}
\maxsecnumdepth{subsection}

%% Profondeur de la ToC
\settocdepth{subsection}
\maxtocdepth{subsection}

%% Style des titres des divisions logiques
\setsecheadstyle{\sffamily\Large\scshape}
\setsubsecheadstyle{\sffamily\large\scshape}

%% style des environnements description
%\renewcommand*{\descriptionlabel}[1]{\hspace\labelsep
%  \normalfont\itshape #1}

%% épigraphes
%\setlength{\epigraphwidth}{0.5\textwidth}
%\epigraphtextposition{flushleftright}


\author{\sffamily\LARGE P244\and\LARGE P272\and\LARGE Pxxx}
\date{\sffamily Week 8, Hilary Term 2018}
\title{\sffamily\scshape\HUGE Statistical Machine Learning\\[.2em]
\LARGE Assessed Practical}



%%% Local Variables:
%%% mode: latex
%%% TeX-master: "report"
%%% End:


\clubpenalty=10000
\widowpenalty=10000
\raggedbottom%

\newcommand{\rsq}{R$^2$~}
\newcommand{\betac}{\widehat{\beta}}
\newcommand{\lkh}{\widehat{L}}
\newcommand{\loglkh}{\ln(\widehat{L})}
\newcommand{\ypred}{\widehat{Y}}
\newcommand{\eg}{\textit{e.g.~}}
\newcommand{\sautLigne}{

  \vspace{0.3cm}}
\newcommand{\expectation}[1]{\mathbb{E}[#1]}
\newcommand{\var}[1]{Var[#1]}
\newcommand{\RP}{\textcolor{red}{\large{\textbf{TBC~}}}}

\newcommand{\fbf}{\texttt{BF}}
\newcommand{\yTrue}{y}
\newcommand{\yPred}{\hat{y}}
\newcommand{\ri}{\rho_i}
\newcommand{\rit}{\rho_{i,t}}
\newcommand{\muit}{\mu_{i,t}}
\newcommand{\thtk}{\widetilde{\theta_k}}
\newcommand{\fhome}{\texttt{Home}}
\newcommand{\fpos}{\texttt{Pos}}

\newcommand{\sq}{$^{\text{2}}$}

\begin{document}

\tightlists%

\maketitle

%% Style de chapitre
%\chapterstyle{}
\setlength{\beforechapskip}{30pt}
\renewcommand{\chapnamefont}{\sffamily\LARGE\scshape}
\renewcommand{\chapnumfont}{\sffamily\LARGE\scshape}
\renewcommand{\chaptitlefont}{\sffamily\LARGE\scshape}

\tableofcontents*

\begin{abstract}
  In this report, we consider the problem of determining the gender of
  an individual in a picture. More precisely, using a set of features
  derived from the original images, we try to fit a machine learning
  model that can accurately classify individuals as male and female,
  and accurately represent its confidence in the classification.

  The performance of the model was evaluated through a Kaggle
  competition, in which we submitted predictions as team ``The Poisson
  Fishermen''.
\end{abstract}

\newpage

\chapter{The data}

The data available are pre-processed data from male and female
pictures. Each picture is represented by a 128-numbers long vector of
features and a label: 0 for male, 1 for female.
    
The labelled training set is composed of 15 000 observations, about
half of which are of female individuals. The recorded features are all
roughly centered and have standard deviations close to $0.9$.
    
Exploratory data analysis does not show any feature particularly
standing out. After a PCA, even the first principal component carries
only around 3\% of the total variance, which suggests variability is
widespread across features. Likewise, the Spearman correlation between
indiviudal features and the labels are rather low (all between -0.3
and 0.3 with most much closer to 0).
    
Despite this lack of interpretability of the features, the data shows
good separability.A simple logistic regression gives 92\% accuracy
when predicting the labels. Likewise, using T-SNE (t-distributed
stochastic neighbor embedding), an unsupervised method that give a 2D
representation of the dataset, clearly separates males and
females. This is promising for the task at hand, since it means that
separation of males and females is quite feasible.
% \includegraphics[tsne]{} %projection du dataset


\section{Evaluating of the model}

To evaluate the performances of our models, the log loss, aka logistic
loss or cross-entropy loss, is used:
\[ -\log(\yTrue | \yPred) = -(\yTrue \log(\yPred) + (1 - \yTrue)
  \log(1 - \yPred))\] with $y$ the true label and $\hat{y}$ the
predicted one. A particularity of this loss is that high confidence in
a wrong prediction is very heavily penalized. This means that not only
classification should be accurate, but predictions must also be
conservative when there is some doubt on the label.

%\includegraphics[logloss]{} % juste le profil de la logloss pour faire un peu de décor

\chapter{Building the model}

\section{Basic classifiers}

Given the good separability of the data, we started by experimenting
with very simple algorithms. Our first successful attempts was the
Quadratic Discriminant Analysis (QDA), which both ran almost
instantaneously and yielded an accuracy over 98\%. K-Nearest
neighbours with 10 neighbours had similar precision.

However, in both cases the log loss remained higher than even the
simple logistic regression (even though its classification performance
was lower). This is because the previous algorithms are not
well-calibrated: the exact value that they give when predicting a
probability is not representative of their true classification
performance. This causes them to be over- or under-confident in their
probability estimates and they are thus heavily penalized by the log
loss metric. Although some methods exist to calibrate such algorithms
(using an isotonic regression for instance), we chose to move on to
algorithms that could directly optimize the right metric.
    
This is the case of the xgboost package, which can perform gradient
boosting on trees using a log loss. This method showed promising
results but failed to consistently bring the average loss below 0.1.

Having gone through those and a number of other standard methods, we
turned to deep learning for comparison. After minimal tuning, a
Multi-Layer Perceptron (MLP) with two hidden layers gave an average
log loss around 0.08. Considering the wide gap between the performance
of this algorithm and the others, we chose to focus the rest of our
study on neural models.

Before moving on to more complex neural network, we tried to make use
of the work mentioned above to derive some useful features for our
final model.
    
\section{Generating new features}

To try and derive interesting features from the data, one possibility
is to use the predictions from lower-performing models as new
covariates for the final model. We chose to use the outputs from the
K-Neighbors and QDA for that purpose: since they are not linearly
derived from the data, they may provide some information that would
take a lot of computation for the neural network to find on its own.
    
We also tried to obtain condensed features with a simple autoencoder
implemented with a multi-layers perceptron. It consists of three
layers: the input and the output are composed of 128 nodes, while the
hidden layer is composed of 32 neurons. By training the NN with
identical input and output, the weights are such that the output of
the hidden layer is an encoding that minimises the loss of
information. Theoretically, this allows for a more informative
representation of the data since the compressed representation from
the hidden layer will reduce the noise of the data and keep its most
informative, so that the output layer can decode this representation
and reconstruct the initial data.

The third path we explored to generate new features was to use
non-linear embeddings of the data in lower-dimensional space. As
previously mentioned, although linear methods such as the PCA and LDA
yielded disappointing separation, more advanced ones such as spectral
embedding and T-SNE.\@ We experimented with those, adding the
projected coordinates as new covariates for the neural model.

All in all, the most successful addition was that of the QDA, which
gave the model a noticeable improvement. Some of the other artificial
features appeared to give a slight boost to the classification but
since the difference was extremely small and the computation of these
features quite costly, we chose to keep only the QDA.\@


\chapter{The final model}

\section{Improving the neural network}

To improve on the simple MLP model's performances, we worked on
building a more advanced neural network using the dedicated PyTorch
framework. In particular a gradient descent with momentum (Adam
algorithm) significantly improved the model.

Quite quickly, it became apparent that using more than one hidden
layer was counter-productive and resulted in strong overfitting. More
generally, the main difficulty was to find ways to improve the fit of
the model without harming its generalization. When training the model
on part of the data and validating on a new set of rows, it became
apparent that the performance of the model depended heavily on the
split between the training and validation data. This high variability
in the performance, associated with the large difference between the
training and validation error, called for regularization.

L2 regularization, implemented through weight decay improved the
overall performance but did not do much to reduce variability. Dropout
--- disabling some of the neurons during training to force
generalization --- had a similar effect. We also experimented with
batch normalization which did not help much.

The end result of our efforts was to bring our public leaderboard
score do 0.074. In local validation, performance varied wildly, from
0.065 to almost 0.08. This is what led us to try and find a way to
reduce variability further.

\section{Preventing overfitting}

The main problem of our base network was large overfitting to the
training data. To prevent this, we tested multiple possible solutions.

The first, obvious method, is to adjust the optimizer and its learning
rate. The Adam method is very common in classification networks, but
we also tested a simple stochastic gradient descent. We determined the
optimal learning rate by cross-validation, taking into account the
number of epochs to train the network. But the learning rate is not
the only relevant hyperparameter: we can also introduce other
regularisation methods, such as momentum for SGD and L2 regularisation
(weight decay) for Adam. Weight decay has given excellent results to
limit overfitting, and has improved the neural network results both in
overall performance and in variability.



\section{Ensembling trees}

% describe the ensembling process, say it brought the error to .071 on the leaderboard
% emphasize that the results are still very unstable + graphicsplots of the residuals
Then the final probability is a weighted average of the output of each
model. The weights are based on the performances of this NN (measured
by cross-validation).  [Add weights formula \RP]

\section{The road to 0.06}
    
\chapter{Conclusion}
%%%%%%%% Dire que si on veut améliorer encore plus il va falloir faire des ensembles avec 
% des modèles différents ou changer d'approche
% Ajouter que de toute façon en vrai on est déjà à genre 99% d'accuracy et 0.999 AUC donc 
% que vu le coût en calcul le plus raisonnable est d'utiliser le MLP (relativement rapide, très bonne performance) ou la QDA (instantané, bonne performance surtout si calibré)




\backmatter%

% \chapter{R code}%
% \label{cha:r-code}

% \inputminted[linenos,stepnumber=5]{r}{../assessed3.R}

%\listoffigures

%% Bibliographie

% \nocite{*}
% \bibliographystyle{apalike}
% \bibliography{report}%
% \label{cha:bibliography}

\end{document}



%%% Local Variables:
%%% mode: latex
%%% TeX-master: t
%%% End:
